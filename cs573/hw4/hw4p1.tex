\documentclass[a4paper,12pt]{article}
\begin{document}
Yongzuan Wu

wu68

cs573

HW 4

Problem 1

(a)

We use the notation for treap analysis in the lecture notes. 

We know that $x_{j}$ is a proper ancestor of node $x_{i}$ iff $x_{j}$ is the minimal element in $X(i,j)$. Also $x_{j}$ is a proper ancestor of $x_{k}$ iff $x_{j}$ is the minimal element in $X(j,k)$. Thus,  $x_{j}$ is a common ancestor of $x_{i}$ and $x_{k}$ iff $x_{j}$ is the minimal element in $X(i,k)$. The probability is $\frac{1}{k-i+1}$.

(b)

Let $x_{j}$ be the deepest common ancestor of $x_{i}$ and $x_{k}$, $i\leq j \leq k$. Then the length of the unique path from $x_{i}$ to  $x_{k}$ is 
\[ $path length$=depth(x_{i})+depth(x_{k})-2depth(x_{j}) \]

We take the expectation value according to the formula from the lecture nodes, i.e. $E[depth(x_{i})]=H_{i}+H_{n-i+1}-2$, then

\begin{array}{lcl}
       

E[$path length between node i and k$ ]&=&E[depth(x_{i})+depth(x_{k})-2depth(x_{j})]\\&=&E[depth(x_{i})]+E[depth(x_{k})]-2E[depth(x_{j})]\\&=&H_{i}+H_{n-i+1}-2+H_{k}+H_{n-k+1}-2

\\&+&\sum_{j=i}^{k}\frac{1}{i-k+1}(H_{j}+H_{n-j+1}-2)$
\end{array}

\end{document}